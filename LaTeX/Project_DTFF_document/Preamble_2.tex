%%
 % Packages 
 %%
 \usepackage{ragged2e} 
 \usepackage{subscript}
 \usepackage[final]{pdfpages}
 \usepackage{epstopdf}
 \includepdfset{pages=-,noautoscale}
\usepackage{url}

\usepackage[export]{adjustbox}			

% Mathe Packages
\usepackage{amsmath,amssymb,epsfig,amsthm,amsfonts,bbm,textcomp,bm}

% Grafiken
\usepackage{graphicx,wrapfig}
\usepackage[font=small,aboveskip=0pt,belowskip=0pt,labelfont=bf]{caption}
\captionsetup[table]{labelsep=period}
\usepackage{caption}
\usepackage{subcaption}

% Sprache Windows / deutsch
\usepackage[ansinew]{inputenc}
\usepackage{babel}				

% Layout
\usepackage{a4}

\usepackage{enumitem}%letters enumaration list 
\usepackage{setspace}
\usepackage{lscape} % Landscape
\usepackage{xcolor} %
\definecolor{darkgreen}{RGB}{0,153,76}
%\usepackage{ragged2e}
\usepackage{parskip}

% Fussnoten
\usepackage[flushmargin,hang,bottom,stable,multiple]{footmisc}
\usepackage{listings} 			% Code und so
%\usepackage[breaklinks]{hyperref}				% aktivieren fuer letzten Durchgang - gibt zwar eine Warnung, doch sind Zeilen schoen gebrochen
\usepackage{hyperref}

% PDF-Informationen
\hypersetup{
    colorlinks=true,
    citecolor=black,
    urlcolor=black,
    linkcolor=black,
    linktoc=page,
    pdftitle={The Value of Dividend Growth Models in the Swiss Stock Market},
    pdfsubject={Master's Thesis in Banking and Finance},
    pdfkeywords={Corporate Payout} {Dividend} {Portfolio Choice} {Swiss Stock Market} 
    pdfauthor={Anja Zgraggen}
}

% Tabellen
\usepackage{longtable} 			% Lange, mehrseitige Tabelle
\usepackage{tabularx}    		% Package used to make variable width-columns, i.e.,
\usepackage{rotating}
\usepackage{array}      				% column widths are changed to fit the maximum width
                            % and text is wrapped nicely.
\usepackage{array,hhline}
\usepackage{booktabs}				% Linien zwischen Zeilen 
\usepackage{multirow}				% Verknüpfe Zellen über Zeilen hinweg
\usepackage{dcolumn}				% Erlaubt variable Orientierung am Decimalpunkt innerhalb einer Zelle
\newcolumntype{.}{D{.}{.}{1}}
\usepackage{makecell}
\linespread{1.1}


% References Teil
\usepackage{natbib}
\usepackage{ifthen} 				% Fuer Probekompilation und Zeitersparnis

% Umgebungsvariablen
\setcounter{secnumdepth}{5} % macht, dass alle Ueberschriften nummeriert werden

% Command Abkürzungen
\newcommand{\be}{\begin{equation}}
\newcommand{\ee}{\end{equation}}
\newcommand{\ben}{\begin*{equation}}
\newcommand{\een}{\end*{equation}}
\newcommand{\E}{\mathbb E}
\makeatletter
\newcommand{\thickhline}{%
	\noalign {\ifnum 0=`}\fi \hrule height 1pt
	\futurelet \reserved@a \@xhline
}
\newcolumntype{"}{@{\hskip\tabcolsep\vrule width 1pt\hskip\tabcolsep}}
\makeatother



\setlength{\abovecaptionskip}{0.1cm}
\setlength{\belowcaptionskip}{-2mm}

% Seiten Layout
\usepackage{float} 					% Um Veränderungen am Textumbruch um Abbildungen zu verhindern
\usepackage[bottom=1.0in,top=1.0in,left=2.0cm,right=2.0cm]{geometry}  
														% Seitenaufbau

% Numbering

\renewcommand{\thetable}{\Roman{table}} 
\makeatletter
% Punkt nach Sektion, aber nicht in Zitaten 
%\renewcommand{\@seccntformat}[1]{{\csname the#1\endcsname}.\hspace{0.5em}}
\long\def\@makefigcaption#1#2{%
 \vskip\abovecaptionskip
 \sbox\@tempboxa{\textbf{#1.} #2}%
 \ifdim \wd\@tempboxa >\hsize
 \textbf{#1.} #2\par
 \else
  \global \@minipagefalse
  \hb@xt@\hsize{\hfil\box\@tempboxa\hfil}%
 \fi
 \vskip\belowcaptionskip} 

% Float fuer Figures
\renewcommand{\figure}{\let\@makecaption\@makefigcaption\@float{figure}}

% Lange Beschreibung fuer Figures
\long\def\@maketblcaption#1#2{%
 \vskip\abovecaptionskip
 \begin{center}\normalsize\textbf{#1. #2}\end{center} %\begin{center}\small\bf#1 \\\normalsize#2\end{center}
 
 \vskip\belowcaptionskip} 

% Float fuer Tables
\renewcommand{\table}{\let\@makecaption\@maketblcaption\@float{table}}
\makeatother


\usepackage[margin=10pt, font=small, labelfont=bf, labelsep=endash]{caption}%To costumize caption captionposition not working
%\usepackage{floatrow} %to have the caption on top



% Footnotes
\makeatletter
\newlength{\myFootnoteWidth}
\newlength{\myFootnoteLabel}
\setlength{\myFootnoteLabel}{1.2em} % <-- can be changed to any valid value
\renewcommand{\@makefntext}[1]{%
 \setlength{\myFootnoteWidth}{\columnwidth}%
 \addtolength{\myFootnoteWidth}{-\myFootnoteLabel}%
 \noindent\makebox[\myFootnoteLabel][r]{\@makefnmark\ }%
 \parbox[t]{\myFootnoteWidth}{#1}%
}
\makeatother


% Abbildung Platz fuellen
\makeatletter
\newcommand\wrapfill{\par
\ifx\parshape\WF@fudgeparshape
\nobreak
\vskip-\baselineskip
\vskip\c@WF@wrappedlines\baselineskip
\allowbreak
\WFclear
\fi
}
\makeatother

% Theoreme